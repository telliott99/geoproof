\documentclass[11pt, oneside]{article} 
\usepackage{geometry}
\geometry{letterpaper} 
\usepackage{graphicx}
	
\usepackage{amssymb}
\usepackage{amsmath}
\usepackage{parskip}
\usepackage{color}
\usepackage{hyperref}

\graphicspath{{/Users/telliott/Dropbox/Github-Math/quickgeo/figures/}{/Users/telliott/Dropbox/Github-Math/figures/}}
% \begin{center} \includegraphics [scale=0.4] {gauss3.png} \end{center}

\title{Thales' theorems}
\date{}

\begin{document}
\maketitle
\Large

%[my-super-duper-separator]

\subsection*{pyramid height}

As we said earlier, Thales was from Miletus and he lived around 600 BC.  Thales is believed to have traveled extensively and was likely of Phoenician heritage.  As you probably know, the Phoenicians were famous sailors who founded many settlements around the Mediterranean.  

They competed with the mainland Greeks and later with the Romans for colonies, and their major city, Carthage, was destroyed much later by the Romans, in the third Punic War.  Hannibal rode his famous elephants over the Alps in the second Punic war.

During his travels, Thales went to Egypt, home to the great pyramids at Giza, which were already ancient then.  They had been built about 2560 BC (dated by reference to Egyptian kings) and were already 2000 years old at that time!

The story is that Thales asked the Egyptian priests about the height of the Great Pyramid of Cheops, and they would not tell him.  So he set about measuring it himself.  He used similar triangles.  I'm sure he wrote down his answer, but I'm not aware that it survives.  The current height is 480 feet.

\begin{center} \includegraphics [scale=0.25] {Thales_theorem_6.png} \end{center}

Plutarch says this:

\begin{quote}the king finds much to admire in you, and in particular he was immensely pleased with your method of measuring the pyramid, because, without making any ado or asking for any instrument, you simply set your walking stick upright at the edge of the shadow which the pyramid cast, and, two triangles being formed by the intercepting of the sun's rays, you demonstrated that the height of the pyramid bore the same relation to the length of the stick as the one shadow to the other.\end{quote}

So the actual method is even cleverer than the diagram shows.

\end{document}
