\documentclass[11pt, oneside]{article} 
\usepackage{geometry}
\geometry{letterpaper} 
\usepackage{graphicx}
	
\usepackage{amssymb}
\usepackage{amsmath}
\usepackage{parskip}
\usepackage{color}
\usepackage{hyperref}

\graphicspath{{/Users/telliott/Dropbox/Github-Math/quickgeo/figures/}{/Users/telliott/Dropbox/Github-Math/figures/}}
% \begin{center} \includegraphics [scale=0.4] {gauss3.png} \end{center}

\title{Eratosthenes}
\date{}

\begin{document}
\maketitle
\Large

%[my-super-duper-separator]

The widely held theory, that the ancient world believed the earth to be flat, is just wrong.  People with any level of sophistication not only knew the earth is roughly spherical but also knew its size within a few percent of the true value.

One likely basis is the false story that Columbus had trouble getting financing for his proposed trip to China because everyone thought he would fall off the edge of the earth.  This was a tall tale invented by Washington Irving, who also made up several remarkable fables about George Washington.

The real reason the Italians and the Portuguese thought Columbus would fail is that they had a pretty good idea of the size of the spherical earth and thus of the distance to China, while the over-optimistic Columbus believed it was about half the true value.  The prospective financiers knew that he was not able to carry the supplies necessary for a trip of this length.

Morris Kline (\emph{Mathematics and the Physical World}) says that the error is due to geographers after Eratosthenes, who reduced the estimated circumference from 24,000 to 17,000 miles.

\subsection*{Eratosthenes}

Views of the Greek philosophers on the earth and its sphericity are detailed here

\url{https://www.iep.utm.edu/thales/#SH8d}

Here is a partial quotation:

\begin{quote}
There are several good reasons to accept that Thales envisaged the earth as spherical. Aristotle used these arguments to support his own view [...] . First is the fact that during a solar eclipse, the shadow caused by the interposition of the earth between the sun and the moon is always convex; therefore the earth must be spherical. In other words, if the earth were a flat disk, the shadow cast during an eclipse would be elliptical. Second, Thales, who is acknowledged as an observer of the heavens, would have observed that stars which are visible in a certain locality may not be visible further to the north or south, a phenomen[on] which could be explained within the understanding of a spherical earth.
\end{quote}

\url{https://en.wikipedia.org/wiki/Eratosthenes}

Eratosthenes (ca. 276 - 195 BCE) measured the circumference of the earth from this observation:  at high noon on June 21st there was no shadow seen at Syene, allegedly from a stick placed vertically in the ground.  Some people say a deep well had the bottom illuminated at midday.

Syene is presently known as Aswan.  It is on the Nile about 150 miles upstream of Luxor, which includes the famous site called the Valley of the Kings.  At 24.1 degrees north latitude, Aswan or Syene is close enough to having the sun directly overhead on June 21.  (The "Tropic of Cancer" is at 23 degrees, 26 minutes north).

\begin{center} \includegraphics [scale=0.6] {aswan.png} \end{center}

Alexandria was a famous center of learning of the ancient world, and Eratosthenes was hired by the pharaoh Ptolemy III to be the librarian in 245 BCE.  Alexandria lies on the Mediterranean some 500 miles north of Syene, and anyone there who was looking could observe that at high noon on June 21st there \emph{was a shadow}.  This shadow Eratosthenes measured to be some 7 degrees and a bit (7 degrees and 10 minutes).

\begin{center} \includegraphics [scale=0.4] {eratosthenes.png} \end{center}

A full 360 degrees divided by 7 degrees and a bit is approximately 50.  So we can calculate on this basis that the circumference of the earth is about $50 \times 500 = 25000$ miles.  That's pretty close to the correct value.

For this calculation, we assume that the sun's rays are effectively parallel (not a bad assumption given a distance of 93 million miles).  Then we just use this:

\begin{center} \includegraphics [scale=0.3] {eratosthenes2.png} \end{center} 

an application of the alternate-interior-angles theorem.

It is curious how the distance from Alexandria to Syene was calculated. 

Kline:

\begin{quote} Camel trains, which usually traveled 100 stadia a day, took 50 days to reach Syene.  Hence the distance was 5000 stadia...It is believed that a stadium was 157 meters.\end{quote}

We obtain
\[ 157 \times 5000 \times 50 = 39,250 \ \text{km} \]

The result is also sometimes given as $5000$ stadia is about $500$ miles, so that gives $25000$ miles for the circumference, when the true value at the equator is 24902 miles (Maor).  That's an error of 1 part in 250.

That's a much better estimate than a method that relies on camels really deserves.  

Some people suspect that the conversion factor from stadia to meters might have been chosen so as to make Eratosthenes estimate look closer than it really was.  According to this

\url{http://www.geo.hunter.cuny.edu/~jochen/gtech201/lectures/lec6concepts/datums/determining%20the%20earths%20size.htm}

The actual angular measurement corresponding to Syene v. Alexandria should have been $7^{\circ} \ 30^{'}$ rather than $7^{\circ} \ 12^{'}$.  The estimate is thus under the true value by $432/450 \approx 1$\%.  I can live with that.

\end{document}
