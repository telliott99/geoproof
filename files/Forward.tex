\documentclass[11pt, oneside]{article} 
\usepackage{geometry}
\geometry{letterpaper} 
\usepackage{graphicx}
	
\usepackage{amssymb}
\usepackage{amsmath}
\usepackage{parskip}
\usepackage{color}
\usepackage{hyperref}

\title{Introduction}
\date{}

\begin{document}
\maketitle
\Large

%[my-super-duper-separator]

This is a concise introduction to geometry, and weighs in at a little over a hundred pages, or it did before I added a couple new things.  It has hand-drawn figures, in homage to Lockhart's  \emph{Measurement} and Acheson's \emph{The Wonder Book of Geometry}.

And although there are some extra words in the first few chapters, effectively all that we do is prove theorems.  By page 34 we have all the big ones.  We are done with the basics of Euclidean geometry by page 120 or so.

However, the last third of the book has significantly more advanced material, so you may find it a bit challenging if you are really just beginning.

There aren't any problems, \emph{per se}.  I suggest that when you finish reading a chapter, go back and prove all the theorems again, without peeking.  Or even better, try to do them before you even read the chapter!

My hope is that this book will bring pleasure to anyone who reads it, but especially those who write out the proofs themselves.

\end{document}
